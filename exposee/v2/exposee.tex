\documentclass{article}
\usepackage[a4paper]{geometry}
\usepackage{cleveref}
\usepackage{tikz}
\usepackage{caption, subcaption}

\newcommand{\qemu}{QEMU}

\title{%
  Master Thesis Proposal\\
  \large Dual rail logic in software as LLVM-IR transformation}
\author{Alexander Schl\"ogl}

\begin{document}
\maketitle

\section{Introduction}
Embedded devices very rarely utilize instruction level parallelism.
Thus, as the power consumption is directly related to the bits in intermediate results that are set to 1, their power consumption directly reflects their computation results without much noise.
If the device is running a cryptographic operation, this can result in a leakage of keys.
This is known as a power analysis side channel attack. \cite{kocher1999differential}

While there exist many different defenses against this, both in software and in hardware, the most versatile of them is Dual-Rail-Logic. \cite{sokolov2005design}
Unlike most other defense mechanisms, Dual-Rail-Logic can be applied to any program.
Unfortunately, using Dual-Rail-Logic requires alterations to the hardware, and almost doubles the required circuitry size, making it unsuitable for small embedded applications like e.g. SmartCards.
In order to create a way of hardening \emph{any} application against side channel attacks, even when there are tight constraints on space, I would like to implement Dual-Rail-Logic in software.

\section{Background}
\label{sec:background}
aoeu

\section{Related Work}
\label{sec:related-work}
aoeu

\section{Intended Methodology}
\label{sec:intended-methodology}
aoeu

\bibliographystyle{plain}
\bibliography{sources.bib}
\end{document}