\section{Power Analysis}
\label{poweranalysis}

In most cases the power consumption during execution is data-dependent.
Setting a bit to 1 requires more power than setting it to 0.
Power consumption is thus directly linked to the \hammingw{} of processed data.
An attacker can then measure the power consumption and make inferences on the data being processed.

Performing \poweranalysis{} requires some setup:
An attacker solders a resistor between the target processor and the ground of its power supply.
She then measures the voltage difference between both ends with an oscilloscope (this voltage is directly proportional to the current flowing through the resistor).
This gives her easy access to the power traces at a high resolution and for every clock cycle.

With these power traces an attacker then has the choice of multiple attack forms of varying complexity.\cite{kocher1998introduction}\cite{brier2004correlation}
The simplest form is \emph{Simple \poweranalysis{}} (SPA), and it involves directly examining the power consumption.
As large control blocks can be identified, a data-dependent control flow can leak information this way.
An example target would be RSA decryption being calculated via the square-and-multiply algorithm.
The difference between the multiply and the square operation is directly observable from a single power trace.
As the order of these operations is linked to the private key, identifying the control flow leaks the private key.
\Cref{fig:spa} shows a trace for square-and-multiply in RSA decryption, including the leaked private key bits.

\begin{figure}[h]
  \centering
  \includegraphics[width=\textwidth]{spa.png}
  \caption{Simple Power Analysis on square-and-multiply RSA\cite{boehme2017netsec}}
  \label{fig:spa}
\end{figure}

The control flow is often not enough to leak the entire secret, and it is very hard to gain information about the actual data from only SPA.
For this a more complex variant of \poweranalysis{} can be used, namely \emph{Differential \poweranalysis{}} (DPA).
DPA requires a large number of traces, with one factor for the power consumption known.
For cryptographic operations this equates a \emph{chosen plaintext} attack scenario.

DPA then attacks the individual bits of the key.
The attacker considers two cases, one for each value of the current key bit.
First she assumes the value of the current key bit is 0.
She then chooses a bitwise operation (e.g. XOR of the plaintext with the key in the first round of AES), and splits the power traces into two sets, based on the value of the target bit in the expected result of this operation.
Next she calculates the mean power consumption of both sets.
As the value of the other bits, as well as other factors for the power consumption, are randomly distributed, calculating the mean will neutralize them.
If her assumption was correct, the difference of both means will exhibit a spike at the time of the chosen operation.
Either way, the value of the current key bit is revealed to her.

\Cref{fig:dpa} shows a typical DPA result with the mean power consumptions of both sets, the difference between the two, and the difference with the Y axis magnified by a factor of 15.
This analysis was performed on the output of the least significant bit after the first S-box substitution in AES.

\begin{figure}[h]
  \centering
  \includegraphics[width=0.8\textwidth]{dpa.png}
  \caption{Difference in means during DPA\cite{kocher2011introduction}}
  \label{fig:dpa}
\end{figure}

Even more higher order information about the key can be found with \emph{Correlational \poweranalysis{}} (CPA).
CPA is the most complex of these attacks, but also offers the best results.
The attacker starts by making a list of candidate values for every byte of the key.
Attacking individual bytes at a time, instead of the whole key, still keeps the required effort feasible.
As she knows which algorithm she is attacking, she knows which operations will take place, and can calculate expected intermediate results based on her chosen plaintexts and key candidates.
She can then calculate the expected power consumptions for these intermediate results.

The attacker can now calculate the correlation between the expected power consumptions for every key byte candidate and the actual power consumption.
The candidate with the highest correlation coefficient is then the most probable value for the current key byte.

\subsection{Defenses against \poweranalysis{} attacks}
\poweranalysis{} works because processed data directly influences the power consumption.
Defenses against this class of attack usually work by either adding additional factors to the power consumption, thus increasing the computational effort required for analysis, or by reducing variances in power consumption altogether.
This reduced variance decreases the information an attacker can gain from the same number of power traces, giving her reduced confidence in her result or requiring her to capture more traces.

Masking\cite{golic2002multiplicative}\cite{coron2000boolean} for example is an algorithm specific defensive measure that adds a third factor to the power consumption by first performing adding a masking value to the plaintext via an invertible operation.
The cryptographic algorithm then works on the masked value, and only in the end unmasks the result.
As such the attacker has to calculate her correlation for each possible combination of key byte and mask value.
This increases the number of traces she needs to capture (to still provide the same confidence in her analysis) and the computation time of her analysis.

Other defensive measures focus on creating a worse signal to noise ratio for the entire power consumption.
One technique that has gained a lot of traction is \dual{}\cite{sokolov2005design}.
It works by calculating the inverse of every intermediate result along with the actual result, thus balancing the number of 1s.
This results in a constant \hammingw{} and therefore a data-independent power consumption.

Unfortunately, \dual{} suffers from multiple engineering problems.
The power required to set the value of a bit to 1 is dependent on properties of the underlying transistors, which are subject to variances in manufacturing.\cite{razafindraibe2006formal}
Minimal differences in clock timings between both paths can also reduce the security of \dual{}\cite{baddam2008path}.
Storing the inverse also requires significantly larger circuitry, doubling the circuit size or more\cite{baddam2008path}.

Even with these caveats, \dual{} has the major advantage that once it is applied, \emph{any} code can be run without modifications while still benefiting from the increased robustness.
