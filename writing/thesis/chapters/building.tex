\section{Build Processes}
\label{build}
Because my thesis project modifies the behavior of the actual compiler and I thus need to control the individual steps of the compilation process, building the test code is a lot more involved than it would be for normal cross-compilation.
Building the pass itself also requires some additional configuration as it needs \llvm{} resources during compilation and it needs to be compatible to my version of the \llvm{} toolchain.

The following sections describe the build setup for the pass and the test code.
They also explain why the additional steps and configurations are necessary, and include code where it benefits understanding.

\subsection{Building the Compiler Pass}
\label{buildpass}
The compiler pass is built using CMake as that makes loading the required parts of \llvm{} very easy.
\Cref{lst:cmakelists} shows the \emph{CMakeLists.txt} for my balancing pass.
The code is based on the template repository provided in \cite{sampson2015llvm}.

\lstinputlisting[caption=CMake configuration for my balancing pass, label=lst:cmakelists, language=CMake]{\project/passes/CMakeLists.txt}

It uses the \emph{find\_package} function of CMake, which sets the locations for definitions, header files, and link directories.
All these are needed to build my pass.
The pass itself is then built as a \emph{MODULE} library, which tells CMake to build a shared library that can be dynamically loaded at runtime by the optimizer.
As the pass is loaded by the optimizer, which is built without run-time type information (RTTI), the pass needs to be built without RTTI as well.

\subsection{Building the Test Code}
\label{buildtest}
As discussed in \Cref{llvm} the \llvm{} compilation process can be split into multiple steps.
I use this feature multiple times in the build process of my test code.
The output of the build process for the RC4 code is shown in \Cref{lst:makefile-output}.

\begin{lstlisting}[caption=Output of the Makefile, label=lst:makefile-output]
arm-none-eabi-as -ggdb  startup.s -o startup.o
clang -target arm-v7m-eabi -mcpu=arm926ej-s -O0 rtlib.c -S -emit-llvm -o rtlib.ll
clang -target arm-v7m-eabi -mcpu=arm926ej-s -O0 program.c -S -emit-llvm -o program.ll
llvm-link rtlib.ll program.ll -S -o linked.ll
opt -load="../../passes/build/libPasses.so" -insert linked.ll -S -o optimized.ll
Balancing module: linked.ll
llc optimized.ll -o optimized.S
arm-none-eabi-as -ggdb  optimized.S -o optimized.o
arm-none-eabi-ld -T startup.ld startup.o optimized.o -o program.elf
arm-none-eabi-objcopy -O binary program.elf program.bin
\end{lstlisting}

Lines 2 and 3 show the translation of the C code into LLVM code, using the Clang\cite{lattner2008llvm} C frontend for \llvm{}.
\emph{Program.c} is the file containing the RC4 code and \emph{rtlib.c} contains the balanced binary operations.
The \emph{-S} flag specifies output to be in human readable \ir{} instead of bytecode, which allows for easier debugging.
The specified \emph{-target} platform and CPU (\emph{-mcpu}) are written into the preamble of the \ir{}, and carried on through the entire toolchain until the compilation into target code on line 7.

Then both \llvm{} files are merged using \emph{llvm-link}, which is simply a concatenation of both files and some reordering.
This merger puts the functions declared in \emph{rtlib.c} in the same module as the target code, and makes them accessible to the compilation pass running on that module.

Line 6 runs the \llvm{} optimizer on the module, loading my balancing pass, which is contained in \emph{libPasses.so}.
The pass is run by issuing the flag assigned to it (\emph{-insert} in this case).
As discussed in \Cref{llvm} both the input and output of the optimizer are \ir{}.
Again the \emph{-S} flag is used for human readable output.
Line 7 shows output of the actual compiler pass.

In line 8 the \ir{} code is compiled into target code, in this case ARM assembly.
The specification of the target platform is taken from the preamble of the \ir{} file, as specified in the frontend call.

The final three steps are handled by the GNU Cross Tools.
First the target code is assembled (line 9) and then it is linked with a prewritten memory map and a fixed startup assembly file (line 10).
The memory map is required due to \qemu{} specifics, as described in \Cref{memory}.
\qemu{} starts execution with the program counter set to address \emph{0x1000} 
Unfortunately, I cannot control the memory layout of the code during and after the compilation process, so I have no guarantee that the \emph{main} main function will land at the desired address.
For this I use a memory map \emph{startup.ld} (as described in \cite{armbare}), which causes the code defined in \emph{startup.s} to be at memory address \emph{0x1000}.
The content of \emph{startup.ld} is shown in \Cref{lst:mmap}.

\begin{lstlisting}[caption=Memory map in \emph{startup.ld}, label=lst:mmap]
ENTRY(_Reset)
SECTIONS
{
 . = 0x10000;
 .startup . : { startup.o(.text) }
 .text : { *(.text) }
 .data : { *(.data) }
 .bss : { *(.bss COMMON) }
 . = ALIGN(8);
 . = . + 0x1000; /* 4kB of stack memory */
 stack_top = .;
}
\end{lstlisting}

The code in \emph{startup.s} then fixes the stack pointer and loads the entry function \emph{c\_entry} in my test code.
