\section{Theory}
\label{theory}

\subsection{Balancing Individual Values}
The first step in finding a balanced arithmetic was finding a balancing scheme for individual values.
While the general shape of the scheme was pretty much clear from the start, the location of $x$ and $\bar{x}$ emerged during my work on the balanced operations.
\Cref{fig:schemes} shows the two schemes that are used in my project.

\begin{figure}[h]
  \centering
  \begin{subfigure}{.49\linewidth}
    \centering
    \tikzbox{scheme1.tex}
    \caption{Balancing Scheme 1}
    \label{fig:scheme1}
  \end{subfigure}
  \begin{subfigure}{0.49\linewidth}
    \centering
    \tikzbox{scheme2.tex}
    \caption{Balancing Scheme 2}
  \end{subfigure}
  \caption{Balancing Schemes}
  \label{fig:schemes}
\end{figure}

In my theoretical work I found balanced operations for both schemes, but in the end decided to use Scheme 1 because it exhibits nicer behavior for shifts, especially rotations.
Both are worth mentioning however, because many of my operations will result in values formatted in Scheme 2 and require explicit transformation.
By finding standardized transformations in both directions I could reuse them in the rest of my arithmetic.

The biggest problem of finding a balanced arithmetic was that $\neg{x \circ y}$ is not $\neg{x} \circ \neg{y}$ ($\circ$ here denotes any operator).
As the ALU cannot execute two different operations on parts of the same register at the same time, there \emph{must} be imbalanced temporary values during execution.
My goal then was to limit the number of these imbalanced values.

\subsection{Balancing Binary Operations}
\label{operations}
After fixing the balancing scheme I started working on finding balanced variants for \ir{}'s binary operators.
As stated above, most operations do not preserve balancedness over all intermediate steps.
They do, however, decrease the signal-to-noise ration for an attacker.
A more detailed analysis can be found in \Cref{balance-eval}.

The notation for the rest of this section is the following:
A single line denotes an intermediate 32bit value, with the individual bytes split by $\bsep{}$.
Notes to the right of $|$ explain how the value in the current line was derived.

\subsubsection{Scheme 1 to Scheme 2}
The transformation from Scheme 1 to Scheme 2 looks as follows:
\begin{align*}
  \binp{1}{0}{\neg{x}}{0}{x}\\
  \btrans{2}{\neg{x}}{\neg{x}}{x}{x}{\%1 \blsl 8}\\
  \btrans{3}{\neg{x}}{0}{0}{x}{\%2 \band \hex{ff0000ff}}
\end{align*}

LSL here stands for logical shift left.

\subsubsection{Scheme 2 to Scheme 1}
The other direction works very similar to the first, and is shown below.
Note that ROR stands for rotational right shift, i.e. the values shifted out on the right are shifted back in on the left.
\begin{align*}
  \binp{1}{\neg{x}}{0}{0}{x}\\
  \btrans{2}{\hex{ff}}{\neg{x}}{0}{x}{\%1 \borr (\%1 \bror 24)}\\
  \btrans{3}{0}{\neg{x}}{0}{x}{\%2 \band \hex{00ff00ff}}
\end{align*}

\subsubsection{ORR}
Before finding a balanced variant of bitwise OR, I needed to find an expression for the inverse of the result.
For this I utilized DeMorgan's law: $\neg{x \lor y} = \neg{x} \land \neg{y}$.
With this equality ORR looks as follows:
\begin{align*}
  \binp{1}{0}{\neg{x}}{0}{x}\\
  \binp{2}{0}{\neg{y}}{0}{y}\\
  \btrans{3}{0}{\neg{x} \borr \neg{y}}{0}{x \borr y}{\%1 \borr \%2}\\
  \btrans{4}{0}{\neg{x} \band \neg{y}}{0}{x \band y}{\%1 \band \%2}\\
  \btrans{5}{\neg{x} \band \neg{y}}{\neg{x} \borr \neg{y}}{x \band y}{x \borr y}{\%3 \borr (\%4 \blsl 8)}\\
  \btrans{6}{\neg{x \borr y}}{0}{0}{x \borr y}{\%5 \band \hex{ff0000ff}}\\
  \btrans{7}{0}{\neg{x \borr y}}{0}{x \borr y}{\trans21(\%6)}
\end{align*}

\subsubsection{AND}
As $\neg{x \land y} = \neg{x} \lor \neg{y}$, AND works almost the same as ORR, but uses different parts of the intermediate results.

\subsubsection{XOR}
XOR is at its base a combination of AND and ORR: $x \oplus y = (\neg{x} \land y) \lor (x \land \neg{y})$.
It is better to create a balanced XOR from scratch, instead of compositioning it from ORR and AND, because both ORR and AND have the same imbalanced intermediate values.

The inverse of the result can be found through repeated application of DeMorgan's law and simplification.
I will skip the details of this simple transformation, and show only the result: $\neg{x \oplus y} = (x \land y) \lor (\neg{x} \land \neg{y})$.

\begin{align*}
  \binp{1}{0}{\neg{x}}{0}{x}\\
  \binp{2}{0}{\neg{y}}{0}{y}\\
  \btrans{3}{\neg{x}}{\neg{x}}{x}{x}{\%1 \borr (\%1 \blsl 8)}\\
  \btrans{4}{y}{\neg{y}}{\neg{y}}{y}{\%2 \borr (\%2 \bror 24)}\\
  \btrans{5}{\neg{x} \band y}{\neg{x} \band \neg{y}}{x \band \neg{y}}{x \band y}{\%3 \band \%4}\\
  \btrans{6}{x \bxor y}{\neg{x \bxor y}}{x \bxor y}{\neg{x \bxor y}}{\%5 \band (\%5 \bror 16)}\\
  \btrans{7}{\neg{x \bxor y}}{x \bxor y}{\neg{x \bxor y}}{x \bxor y}{\%6 \bror 8}\\
  \btrans{8}{\neg{x \bxor y}}{0}{0}{x \bxor y}{\%7 \band \hex{ff0000ff}}\\
  \btrans{9}{0}{\neg{x \bxor y}}{0}{x \bxor y}{\trans21 (\%8)}
\end{align*}

\subsubsection{ADD}
For the inverse of arithmetic operations I utilized the definition of the negation in 2s complement: $-x = \neg{x} + 1$.
This also means that $\neg{x} = -x - 1$ and therefore:
\begin{equation*}
  \neg{x + y} = - (x + y) - 1 = - x - y - 1 = \neg{x} + 1 + \neg{y} \cancel{+ 1} \cancel{- 1} = \neg{x} + \neg{y} + 1
\end{equation*}

Using associativity of addition the balanced variant of ADD looks like the following:
\begin{align*}
  \%1 &= 0 && \bsep \neg{x} &&\bsep 0 &&\bsep x      &&\\
  \%2 &= 0 && \bsep \neg{y} &&\bsep 0 &&\bsep y      &&\\
  \%3 &= 0 && \bsep \neg{x}+1 &&\bsep 0 &&\bsep x    &&\;|\ \%1 + \hex{00010000}\\
  \%4 &= c && \bsep \neg{x+y} &&\bsep c' &&\bsep x+y &&\;|\ \%3 + \%2\\
  \%5 &= 0 && \bsep \neg{x+y} &&\bsep 0 &&\bsep x+y  &&\;|\ \%4 \land \hex{00ff00ff}
\end{align*}
Both $c$ and $c'$ denote possible carry values that need to be filtered.

\subsubsection{SUB}
For subtraction I again use the definition of 2s complement, giving me the following for the inverse result:
\begin{equation*}
  \neg{x-y} = - (x-y) - 1 = y - x - 1 = y + (-x -1) = y + \neg{x} = \neg{x} + y
\end{equation*}
Applying the same definition to the regular result yields
\begin{equation*}
  x-y = x + \neg{y} + 1
\end{equation*}
resulting in a quick and convenient balanced subtraction:
\begin{align*}
  \binp{1}{0}{\neg{x}}{0}{x}\\
  \binp{2}{0}{\neg{y}}{0}{y}\\
  \btrans{3}{0}{y}{0}{\neg{y}}{\%2 \bror 16}\\
  \btrans{4}{0}{y}{c}{\neg{y}+1}{\%3 + \hex{00000001}}\\
  \btrans{5}{c'}{\neg{x}+y}{c''}{x+\neg{y}+1}{\%1 + \%4}\\
  \btrans{6}{0}{\neg{x-y}}{0}{x-y}{\%5 \band \hex{00ff00ff}}
\end{align*}

\subsubsection{MUL}
The inverse result of multiplication can be calculated as follows:
\begin{equation*}
  \neg{x \cdot y} = -(x \cdot y) - 1 = (-x) \cdot y - 1 = (\neg{x} + 1) \cdot y = \neg{x} \cdot y + y - 1
\end{equation*}

Which gives us the following balanced multiplication:
\begin{align*}
  \binp{1}{0}{\neg{x}}{0}{x}\\
  \binp{2}{0}{\neg{y}}{0}{y}\\
  \btrans{3}{\neg{y}}{0}{0}{y}{\trans21(\%2)}\\
  \btrans{4}{c}{\neg{x}\cdot y}{c'}{x \cdot y}{\%1 \cdot \%3}\\
  \btrans{5}{c''}{\neg{x \cdot y} +1}{c'}{x \cdot y}{\%4 + (\%2 \blsl 16)}\\
  \btrans{6}{c'''}{\neg{x \cdot y}}{c'}{x \cdot y}{\%5 + \hex{00ff0000}}\\
  \btrans{7}{0}{\neg{x \cdot y}}{0}{x \cdot y}{\%6 \band \hex{00ff00ff}}
\end{align*}

\subsubsection{DIV and REM}
I used repeated balanced subtraction for DIV and REM operations.
The code was written in C and can be found in the git of my thesis\cite{git}.
%TODO: include?

\subsubsection{Shifting}
While performing logical shifts, I need to ensure that the correct bits are pushed in.
When 0s are shifted in for $x$ I have to shift in 1s for $\neg{x}$, and vice versa.
This is done by ORRing the target value with \hex{ff000000} or \hex{0000ff00}, as needed.
The shifting is performed normally and the result is then AND filtered with \hex{00ff00ff} to comply with Scheme 1 again.

\subsection{Testing for Correctness}
Before I started implementing my balancing pass I wanted to verify the correctness of my arithmetic.
For this purpose I wrote python code to calculate all operations step by step while saving the intermediate results.
\Cref{lst:multiop} shows the intermediate steps for multiplication.

\begin{lstlisting}[language=python, caption=Step-by-step execution of balanced multiplication, label=lst:multiop]
m = MultiStepOperation([
    Convert_1_2(1), #2
    BinaryOperation(0,2, lambda x,y: (x*y) & 0xffffffff), #3 the AND is required due to python's arbitrary precision integers
    BinaryOperation(3,1, lambda x,y: x + (y << 16)), #4
    UnaryOperation(4, lambda x: x + 0x00ff0000), #5
    UnaryOperation(5, lambda x: x & 0x00ff00ff), #6
])
\end{lstlisting}

The \emph{Unary-} and \emph{BinaryOperation} classes take the indices of the layers to operate on (0 and 1 are the inputs, all others are intermediate values), as well as the operation in form of a lambda.
Executing the \emph{MultiStepOperation} will then execute all lambdas in order and store the intermediate results in \emph{numpy} arrays.
Correctness is then tested by checking if all final results are equal to the output of a function to compare to ($x \cdot y$ in this case).

%% \begin{lstlisting}[language=python, caption=Testing correctness of balanced operation, label=lst:correctness]
%% def unbalanceScheme1(x):
%%     return x & word_filter

%% vUS1 = np.vectorize(unbalanceScheme1)


%% def testCorrectness(self, equivalentUnbalancedFunction):
%%     uResults = vUS1(self.results[-1])
%%     incorrectResults = {}
%%     for i in range(word_max):
%%         for j in range(word_max):
%%             r = equivalentUnbalancedFunction(i, j)
%%             r &= word_filter
%%             if r != uResults[i*word_max + j]:
%%                 incorrectResults[(i,j)] = (r, uResults[i*word_max + j])
%%     return incorrectResults
%% \end{lstlisting}

\subsection{Evaluating the Balancedness}
\label{balance-eval}
Balancedness of my operations is evaluated using the same python code.
As all intermediate results are stored during evaluation I can easily calculate the distribution of their \hammingw s, as shown in \Cref{fig:mult}.
I used these histograms to check if operations needed improvement, and if that was the case, I tried to find a different, more balanced way of performing them.

\begin{figure}[h]
  \centering
  \includegraphics[width=\textwidth]{multiplication.png}
  \caption{Histogram of \hammingw s of direct balanced multiplication}
  \label{fig:mult}
\end{figure}

While \Cref{fig:mult} shows imbalanced values in the intermediate steps, it performed faster and better than multiplication via repeated addition.
\Cref{fig:mult-comparison} shows an evaluation of both variants, evaluated over the multiplications of all possible 8bit factors.

\begin{figure}[hp]
  \centering
  \begin{subfigure}[b]{0.49\textwidth}
    \input{tikz/mult-loop.tex}
    \caption{Addition-loop multiplication}
  \end{subfigure}
  \begin{subfigure}[b]{0.49\textwidth}
    \input{tikz/mult-direct.tex}
    \caption{Direct multiplication}
  \end{subfigure}

  \begin{subfigure}[b]{\textwidth}
    \begin{tikzpicture}
\begin{axis}[
    ybar=2*\pgflinewidth,
    bar width=0.00725\textwidth,
    enlargelimits=0.05,
    xtick={0,8,16,24,32},
    width=\textwidth,
    xlabel={\hammingw{}},
    ylabel={Fraction of Occurences},
]
\addplot[pantone289,fill=pantone289!40] table [x=i, y=unbalanced-scaled, col sep=comma] {data/rc4.csv};
\addplot[pantone144,fill=pantone144!40] table [x=i, y=balanced-scaled, col sep=comma] {data/rc4.csv};
\legend{Unbalanced,Balanced}
\end{axis}
\end{tikzpicture}

    \caption{Scaled Hamming weight histograms for multiplication variants}
  \end{subfigure}
  \caption{Hamming weight histograms for direct and addition-loop multiplication}
  \label{fig:mult-comparison}
\end{figure}

