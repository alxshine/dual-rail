\chapter{Evaluation Methodology}
\label{evaluation}
To test the effectiveness of my balancing I compared balanced code with regular code.
In order to decrease the turnaround time during development I decided to run the code in an emulator, as opposed to on actual hardware.
This also increases the detail at which I can examine the results.
Instead of trying to perform a full \poweranalysis{} attack on actual hardware, I tried to add code to the \qemu{} emulator that generates simulated power traces during execution, as well as a log of executed instructions, similar to output of the \emph{gdb} debugger.
These power traces simulate an attacker that can measure the \hammingw{} of every single instruction result without noise.

Unfortunately, performing the evaluation inside \qemu{} has the disadvantage that it does not accurately simulate the hardware capabilities of ARM, like inline computation of addresses and shifting of operands, and thus adds additional imbalances to the power traces.
For this reason I moved the evaluation to gdb, which allows me to evaluate results of every ARM assembly instruction, instead of every instruction in the intermediate representation of \qemu{}.
Even though I used gdb in the end, I included my work on \qemu{} in this section, as it gives some interesting insights into its internal workings and justifies why it is unsuitable for my evaluation.

\section{\qemu{}}
\qemu{} is a generic and open source machine emulator and virtualizer.\cite{bellard2005qemu}
While it can be used as a full fledged virtualization environment and sandbox, it can also emulate different processor architectures for programs without first emulating an OS.
This process, called \emph{bare-metal emulation}, allows me to evaluate the performance of my thesis project on a simulated ARM processor running on my computer.

During the execution of emulated code, so-called guest code, it can also provide a GDB server, allowing for remote debugging of code running inside \qemu{}.
Unfortunately this debugging has to be done in ARM assembly, as all C debugging information has been lost during the build process.

\subsection{Memory Layout of \qemu{} Kernels}
\label{memory}
Even with bare-metal emulation, \qemu{} still takes its input as a kernel.
Due to this, it starts execution at address \hex{1000}, as everything before that address is usually reserved for interrupt handling.
This requires some additional setup in my build process (see \Cref{buildtest}).

\subsection{Extending \qemu{}}
To perform any evaluation I first needed to add code to \qemu{} that computes \hammingw{} histograms during execution.
Doing this proved harder than expected, due to optimizations that happen during execution of guest code.

\qemu{} does not simply interpret the guest code in a simulated processor.
Instead it translates the machine code for the guest platform into machine code for the host platform, and places that ``patched'' machine code in memory.
A second executor thread then runs that code as it becomes available.

This translation backend is called the Tiny Code Generator (TCG), which not only performs the translation but also some optimizations.
Instrumenting \qemu{} for analysis is hard due to the fact that the TCG works through multiple layers of indirection, utilizing both helper functions and preprocessor macros, some of which are defined in different files depending on the host architecture (the specific definition file is chosen while building \qemu{}).

This multi threaded approach also makes examining values during execution very hard.
The executor thread does not know what code it is executing, it only has a pointer (the simulated program counter) to the next instruction or the next basic block.
The TCG on the other hand knows which operations are being executed, but it does not know the values of the operands.
It also has no way of accessing these values as they might not even be computed yet.
So short of either parsing the memory at the simulated program counter or writing a symbolic execution engine for TC (essentially replacing \qemu{}), this emulation model cannot be used for my thesis.

Luckily, \qemu{} also offers emulation via the TC Interpreter (TCI).
The TCI does not transform TC into host assembly, and instead simulates a guest CPU in software.
During this simulation ALU operations are handled by special functions, and their return values are then stored in the simulated registers.
The call sites of these functions are where my evaluation code is located, generating power consumption traces and logs of executed instructions.

\section{Evaluation with GDB}
Unfortunately, even TCI does not fully emulate the guest CPU.
Instead it simulates a general CPU that executes TCG intermediate code.
For my evaluation this means that address calculations, which happen inline on ARM platforms, happen explicitly and add additional unbalanced results to my power traces.
Because of this I decided to use gdb for my evaluation.

I wrote a small script that automatically steps through every assembly instruction and afterwards prints the values in all general purpose registers, utilizing the logging feature of gdb to write a transcript of instructions and register states.
The register states are then filtered out of the transcript with some regex matching.
Finally, I wrote a small python script that detects which register value has changed after the instruction, if any, and generates a power trace from these changes.
This allows me even more detailed evaluation than just the power traces alone.
From the transcript I also extracted the corresponding line in the source assembly file for every point in the power trace.
With both the value and the location in the assembly file I can examine exactly when and where imbalanced values are stored in registers, as shown in \Cref{evaluation}.

\section{Attacker Model}
The traces generated by gdb are perfectly accurate and free of noise.
They also have exactly one trace point per instruction, resulting in a very powerful attacker.
Any statistical analysis she makes will have results with high confidence, as they are not influenced by noise.
The only disadvantage such an attacker has is that she is confined to our power model.
So, even though her measurements are exact, she is limited to the sources of leakage I try to mitigate, and does not have access to the power consumption of any other components.

\section{Test Code}
The main test program for my balancing pass was an implementation of AES~\cite{daemen2013design} for embedded devices.
This implementation~\cite{tinyaes} is tested against the test vectors for AES as specified by NIST~\cite{dworkin2001recommendation}.
I did make some changes to the code, moving as many variables to the stack as possible, in order to maximize the balancing my pass can perform.
During development, I also used RC4~\cite{rc4}, SHA-3~\cite{bertoni2013keccak} and various small programs to test correctness of my pass, but none of them as extensively as AES.
AES's small key sizes (128 to 256 bits), and efficient software implementations make it suitable for embedded applications, which is why I chose it as my main test program.

Using gdb for the evaluation has the disadvantage that execution takes longer than on \qemu{} alone, due to the constant hand-off between gdb and the emulated CPU.
On my laptop running AES encryption through \qemu{} with my gdb script took around 30 minutes, generating 2 million trace points in the process.
In order to both reduce the evaluation time, and provide less cluttered plots in \Cref{results}, I decided to focus solely on the encryption, which only takes around one sixth of the total execution time for the implementation I used.

\subsection{Generating different plaintexts}
In order to evaluate \hammingw{} differences between executions, these executions must use different plaintexts.
I did not want to try and get I/O working during execution, so I compiled different executables with different plaintexts.
The plaintexts bytes in the program are taken as \texttt{\#define}s, which are filled with random numbers in the Makefile, using the shell builtin \texttt{\$RANDOM} and the modulo operator.
Generating multiple binaries with different plaintexts was then accomplished by simply executing make multiple times.

\subsection{Building the Test Code}
\label{buildtest}
Because I need to load my plugin and my balanced arithmetic functions during the optimization step, and because of the memory layout in \qemu{} kernels, the build process is a lot more involved than in normal cross compilation.
As discussed in \Cref{llvm} the \llvm{} compilation process can be split into multiple steps.
I use this feature multiple times in the build process of my test code.
The output of the build process for the AES code is shown in \Cref{lst:makefile-output}.

\begin{lstlisting}[caption=Output of the Makefile, label=lst:makefile-output]
arm-none-eabi-as -ggdb  startup.s -o startup.o
clang -target arm-none-eabi -mcpu=arm926ej-s -O0 tinyAES.c -S -emit-llvm -DC0=127 -DC1=90 -DC2=197 -DC3=51 -DC4=205 -DC5=78 -DC6=146 -DC7=15 -DC8=146 -DC9=22 -DC10=40 -DC11=156 -DC12=42 -DC13=219 -DC14=211 -DC15=147 -o tinyAES.ll
clang -target arm-none-eabi -mcpu=arm926ej-s -O0 rtlib.c -S -emit-llvm -DC0=144 -DC1=118 -DC2=160 -DC3=148 -DC4=68 -DC5=140 -DC6=15 -DC7=107 -DC8=60 -DC9=91 -DC10=15 -DC11=52 -DC12=231 -DC13=252 -DC14=96 -DC15=219 -o rtlib.ll
clang -target arm-none-eabi -mcpu=arm926ej-s -O0 program.c -S -emit-llvm -DC0=165 -DC1=14 -DC2=20 -DC3=128 -DC4=46 -DC5=7 -DC6=35 -DC7=189 -DC8=242 -DC9=253 -DC10=57 -DC11=19 -DC12=180 -DC13=205 -DC14=74 -DC15=103 -o program.ll
arm-none-eabi-as -ggdb  u_startup.s -o u_startup.o
echo 110,252,242,185,151,32,218,85,233,179,98,188,96,77,30,67 >> plaintexts.txt
llvm-link tinyAES.ll rtlib.ll program.ll -S -o linked.ll
llvm-link tinyAES.ll program.ll -S -o u_linked.ll
opt u_linked.ll -S -o u_optimized.ll
opt -load="../../passes/build/libPasses.so" -balance linked.ll -S -o optimized.ll
llc -O0 u_optimized.ll -o unbalanced.S
llc -O0 optimized.ll -o optimized.S
arm-none-eabi-as -ggdb  unbalanced.S -o unbalanced.o
arm-none-eabi-as -ggdb  optimized.S -o optimized.o
arm-none-eabi-ld -T u_startup.ld u_startup.o unbalanced.o -o unbalanced.elf
arm-none-eabi-ld -T startup.ld startup.o optimized.o -o balanced.elf
\end{lstlisting}

Lines 2, 3, and 4 show the translation of C code into LLVM code, using the Clang\cite{lattner2008llvm} C frontend for \llvm{}.
\emph{Program.c} is the file containing the main function, \emph{rtlib.c} contains the balanced binary operations, and \emph{tinyAES.c} contains the AES implementation.
The \emph{-S} flag specifies output to be in human readable \ir{} instead of bytecode, which allows for easier debugging.
The specified \emph{-target} platform and CPU (\emph{-mcpu}) are written into the preamble of the \ir{}, and carried on through the entire toolchain until the compilation into target code on lines 11 and 12.
Also visible are the \texttt{\#defines} used for the plaintext.

The \ir{} files are merged in lines 7 and 8 using \emph{llvm-link}.
This merger puts the functions from all files in the same module as the target code, and makes them accessible to the compilation pass running on that module.
This step is run twice, generating two different \emph{.ll} files.
Only one \emph{.ll} file is going to be balanced with my optimizer pass, allowing comparison of balanced and unbalanced code on the same plaintexts.

In line 9 the \llvm{} optimizer is run without my pass on the unbalanced code, and in line 10 the optimizer is run with my pass.
The optimization pass is contained in \emph{libPasses.so}, and called via the \emph{-balance} flag.
As discussed in \Cref{llvm} both the input and output of the optimizer are \ir{}.
Again the \emph{-S} flag is used for human readable output.

In lines 11 and 12 the \ir{} code is compiled into target code, in this case ARM assembly.
Note that the \texttt{-O0} flag is issued, which should in theory disable all optimizations in the compiler backend.
The specification of the target platform is taken from the preamble of the \ir{} file, as specified in the frontend call.

The final three steps are handled by the GNU Cross Tools.
First the target code is assembled (lines 13 and 14) and then it is linked with a pre-written memory map and a fixed startup assembly file (lines 15 and 16).
The memory map is required due to \qemu{}'s memory layout, as discussed in \Cref{memory}.
\qemu{} starts execution with the program counter set to address \emph{0x1000}.
Unfortunately, I cannot control the memory layout of the code during and after the compilation process, so I have no guarantee that the \emph{main} function will land at the desired address.
For this I use a memory map \emph{startup.ld} (as described in \cite{armbare}), which causes the code defined in \emph{startup.s} to be at memory address \emph{0x1000}.
The content of \emph{startup.ld} is shown in \Cref{lst:mmap}.

\begin{lstlisting}[caption=Memory map in \emph{startup.ld}, label=lst:mmap]
ENTRY(_Reset)
SECTIONS
{
 . = 0x10000;
 .startup . : { startup.o(.text) }
 .text : { *(.text) }
 .data : { *(.data) }
 .bss : { *(.bss COMMON) }
 . = ALIGN(8);
 . = . + 0x1000; /* 4kB of stack memory */
 stack_top = .;
}
\end{lstlisting}

The code in \emph{startup.s} then fixes the stack pointer and loads the entry function \emph{c\_entry} in my test code, as shown in \Cref{lst:startup}.

\begin{lstlisting}[caption=Startup assembly code, label=lst:startup]
.global _Reset
_Reset:
 LDR sp, =stack_top
 BL balanced_c_entry
 B .
\end{lstlisting}

My pass creates new names for the balanced functions it generates.
These new names are simply the old ones, with a \texttt{balanced\_} prefix added to them.
Due to this the balanced and unbalanced code require different startup assembly files, \emph{startup.s} and \emph{u\_startup.s} respectively.
The startup files are assembled in line 1 for the balanced version, and in line 5 for the unbalanced version.

