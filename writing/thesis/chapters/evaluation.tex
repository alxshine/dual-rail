\section{Evaluation}
\label{evaluation}
To test the effectiveness of my balancing I compared balanced code with regular code.
In order to decrease the turnaround time during development I decided to run the code in an emulator, as opposed to on actual hardware.
This also reduces the amount of effort for evaluation.

Instead of running a full \poweranalysis{} attack on actual hardware I added code to the \qemu{} emulator that generates histograms of \hammingw{}s during execution.
This essentially simulates a scenario where an attacker has precise enough instruments to measure the \hammingw{} of every register during every clock cycle.
Such a scenario would equate an extremely powerful attacker, and any robustness improvements for this scenario indicate improvements in real-world scenarios with reasonable confidence.

\subsection{\qemu{}}
\qemu{} is a generic and open source machine emulator and virtualizer.\cite{bellard2005qemu}
While it can be used as a full fledged virtualization environment and sandbox, it can also emulate different processor architectures for programs without first emulating an OS.
This process, called \emph{bare-metal emulation}, allows me to evaluate the performance of my thesis project on a simulated ARM processor running on my computer.

During the execution of emulated code, so-called guest code, it can also provide a GDB server, allowing for remote debugging of code running inside \qemu{}.
Unfortunately this debugging has to be done in ARM assembly, as all C debugging information has been lost during the build process.

\subsubsection{Memory Layout of \qemu{} Kernels}
\label{memory}
Even with bare-metal emulation, \qemu{} still takes its input as a kernel.
Due to this, it starts execution at address \hex{1000}, as everything before that address is usually reserved for interrupt handling.
This requires some additional setup in my build process (see \Cref{buildtest}).

\subsubsection{Extending \qemu{}}
To perform any evaluation I first needed to add code to \qemu{} that computes \hammingw{} histograms during execution.
Doing this proved harder than expected, due to optimizations that happen during execution of guest code.

\qemu{} does not simply interpret the guest code in a simulated processor.
Instead it translates the machine code for the guest platform into machine code for the host platform, and places that ``patched'' machine code in memory.
A second executor thread then runs that code as it becomes available.

This translation backend is called the Tiny Code Generator (TCG), which not only performs the translation but also some optimizations.
Instrumenting \qemu{} for analysis is hard due to the fact that the TCG works through multiple layers of indirection, utilizing both helper functions and preprocessor macros, some of which are defined in different files depending on the host architecture (the specific definition file is chosen while building \qemu{}).
As documentation is also sparse, finding a good place to put my evaluation code required a lot of time and effort.

Even after understanding all the parts of \qemu{}'s way of emulating code, I was left with a problem.
The executor thread does not know what code it is executing, it only has a pointer (the simulated program counter) to the next instruction or the next basic block.
The TCG on the other hand knows which operations are being executed, but it does not know the values of the operands.
It also has no way of accessing these values as they might not even be computed yet.
So short of either parsing the memory at the simulated program counter or writing a symbolic execution engine (essentially replacing \qemu{}) I did not know how to proceed.

Luckily, \qemu{} offers emulation via the TCG Interpreter (TCI).
The TCI does exactly what I was looking for in the first place, i.e. emulating the guest processor in C.
I then placed my instrumentation code in the operator functions of the TCI, generating a histogram of \hammingw{}s during the execution.

\subsection{Test Code}
I tested my balancing with RC4\cite{rc4} and AES\cite{daemen2013design}.
RC4 used to be the industry standard for symmetric encryption, and has an extremely small codebase.
AES is the current symmetric encryption standard.
Both algorithms fit my target use case of encryption on an embedded device.

\subsubsection{Building the Test Code}
\label{buildtest}
Because I need to load my plugin and my balanced arithmetic functions during the optimization step, and because of the memory layout in \qemu{} kernels, the build process is a lot more involved than in normal cross compilation.

As discussed in \Cref{llvm} the \llvm{} compilation process can be split into multiple steps.
I use this feature multiple times in the build process of my test code.
The output of the build process for the RC4 code is shown in \Cref{lst:makefile-output}.

\begin{lstlisting}[caption=Output of the Makefile, label=lst:makefile-output]
arm-none-eabi-gcc --specs=nosys.specs program.c -o program_unbalanced.bin
arm-none-eabi-as -ggdb  startup.s -o startup.o
clang -target arm-v7m-eabi -mcpu=arm926ej-s -O0 rtlib.c -S -emit-llvm -o rtlib.ll
clang -target arm-v7m-eabi -mcpu=arm926ej-s -O0 program.c -S -emit-llvm -o program.ll
llvm-link rtlib.ll program.ll -S -o linked.ll
opt -load="../../passes/build/libPasses.so" -insert linked.ll -S -o optimized.ll
Balancing module: linked.ll
llc optimized.ll -o optimized.S
arm-none-eabi-as -ggdb  optimized.S -o optimized.o
arm-none-eabi-ld -T startup.ld startup.o optimized.o -o program.elf
arm-none-eabi-objcopy -O binary program.elf program.bin
\end{lstlisting}

Line 1 shows the compilation of the unbalanced version that I use for comparison.
This version is compiled using only the \crossgcc{} compiler.
Lines 3 and 4 show the translation of the C code into LLVM code, using the Clang\cite{lattner2008llvm} C frontend for \llvm{}.
\emph{Program.c} is the file containing the RC4 code and \emph{rtlib.c} contains the balanced binary operations.
The \emph{-S} flag specifies output to be in human readable \ir{} instead of bytecode, which allows for easier debugging.
The specified \emph{-target} platform and CPU (\emph{-mcpu}) are written into the preamble of the \ir{}, and carried on through the entire toolchain until the compilation into target code on line 8.

Then both \llvm{} files are merged using \emph{llvm-link}, which is simply a concatenation of both files and some reordering.
This merger puts the functions declared in \emph{rtlib.c} in the same module as the target code, and makes them accessible to the compilation pass running on that module.

Line 6 runs the \llvm{} optimizer on the module, loading my balancing pass, which is contained in \emph{libPasses.so}.
The pass is run by issuing the flag assigned to it (\emph{-insert} in this case).
As discussed in \Cref{llvm} both the input and output of the optimizer are \ir{}.
Again the \emph{-S} flag is used for human readable output.
Line 7 shows output of the actual compiler pass.

In line 8 the \ir{} code is compiled into target code, in this case ARM assembly.
The specification of the target platform is taken from the preamble of the \ir{} file, as specified in the frontend call.

The final three steps are handled by the GNU Cross Tools.
First the target code is assembled (line 9) and then it is linked with a prewritten memory map and a fixed startup assembly file (line 10).
The memory map is required due to \qemu{} specifics, as described in \Cref{memory}.
\qemu{} starts execution with the program counter set to address \emph{0x1000} 
Unfortunately, I cannot control the memory layout of the code during and after the compilation process, so I have no guarantee that the \emph{main} main function will land at the desired address.
For this I use a memory map \emph{startup.ld} (as described in \cite{armbare}), which causes the code defined in \emph{startup.s} to be at memory address \emph{0x1000}.
The content of \emph{startup.ld} is shown in \Cref{lst:mmap}.

\begin{lstlisting}[caption=Memory map in \emph{startup.ld}, label=lst:mmap]
ENTRY(_Reset)
SECTIONS
{
 . = 0x10000;
 .startup . : { startup.o(.text) }
 .text : { *(.text) }
 .data : { *(.data) }
 .bss : { *(.bss COMMON) }
 . = ALIGN(8);
 . = . + 0x1000; /* 4kB of stack memory */
 stack_top = .;
}
\end{lstlisting}

The code in \emph{startup.s} then fixes the stack pointer and loads the entry function \emph{c\_entry} in my test code, as shown in \Cref{lst:startup}.

\begin{lstlisting}[caption=Startup assembly code, label=lst:startup]
.global _Reset
_Reset:
 LDR sp, =stack_top
 BL balanced_c_entry
 B .
\end{lstlisting}
