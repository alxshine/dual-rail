\section{Arithmetic}
The first step in finding a balanced arithmetic was finding a scheme for the balancing of the individual values.
While the general shape of the scheme was pretty much clear from the start, the location of $x$ and $\bar{x}$ emerged during my work on the balanced operation.
\Cref{fig:schemes} shows the two schemes that are used in my project.

\begin{figure}[h]
  \centering
  \begin{subfigure}{.49\linewidth}
    \centering
    \tikzbox{scheme1.tex}
    \caption{Balancing Scheme 1}
    \label{fig:scheme1}
  \end{subfigure}
  \begin{subfigure}{0.49\linewidth}
    \centering
    \tikzbox{scheme2.tex}
    \caption{Balancing Scheme 2}
  \end{subfigure}
  \caption{Balancing Schemes}
  \label{fig:schemes}
\end{figure}

In my theoretical work I found balanced operations for both schemes, but in the end decided to use Scheme 1 because it exhibits nicer behaviour for shifts, especially rotations.
Both are worth mentioning however, because many of my operations will result in values formatted in Scheme 2 and require explicit transformation.
By using the barrel shifter on the ARM architecture this transformation can be done in two cycles without unbalanced intermediate values, so the cost of this transformation is minimal.

\subsection{Finding Balanced Operations}
\label{operations}
aoeu

\subsubsection{AND}
aoeu

\subsection{Testing for Correctness}
aoeu

\subsection{Evaluating the Balancedness}
aoeu
