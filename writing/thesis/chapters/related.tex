\section{Related Work}
\label{related}
I considered related work from three major areas:
Software based \poweranalysis{} defenses that can in theory be applied to any algorithm, work on \dual{}, and security related research utilizing \llvm{}.

First introduced by Messerges\cite{messerges2000securing}, masking is in theory applicable to any algorithm.
In his paper he identified the core operations in the candidates algorithms for AES.
He then devised a way to apply a masking to the input of these operations, in such a way that no intermediate value is stored in memory without being masked.
For example, his mask for boolean operations was applied via XOR: $x_{masked} = x \oplus r$, where $r$ is a randomly selected mask.
Boolean operations can then be applied to $x_{masked}$, and intermediate results do not directly give an attacker information about any secrets in the algorithm.
An attacker can still try to extract both $r$ and the secret key at the same time, however this exponentially increases her required effort.

As not only boolean operations are used, an additional masking scheme for arithmetic operations is required.
Additionally, a conversion method between both masking variants is needed, and this transformation can at no point store $x$ as an intermediate value.
Messerge's approach stores either $x$ or $\neg{x}$, depending on a random value.

While this provides adequate robustness against DPA attacking a single bit of the key, it is insufficient for attacks on multiple bits, as shown by Coron et. al.\cite{coron2000boolean}.
By attacking two bits at the same time, they could find whether both bits were identical or not, and thus gradually reduce the number of possible keys.

Other work has been done to avoid leaks such as this\cite{akkar2001implementation}\cite{rivain2010provably}, however all of these approaches are currently algorithm specific, with most work being targeted at AES.
The decomposition from \cite{messerges2000securing} could be used to automate this masking process for algorithms utilizing these basic operators, and the masking could then be applied automatically during compilation.

Another software approach is inspired by secret sharing.
Goubin et. al.\cite{goubin1999and} used their so-called ``Duplication'' method to defend DES against \poweranalysis{}.
They did this by splitting the input $v$ into multiple parts $v_1, ..., v_n$ using XOR, then performing all operations on the individual parts, and finally recombining them into the encrypted $v'$.
This works because all operations of DES are invariant to the XOR split performed on $v$.

The approach was generalized by Chari et. al.\cite{chari1999towards} using the identity $v=f(v_1,v_2,...,v_n)$.
They state that this secret sharing method can then be applied to any algorithm, as long as no operation invalidates this identity.
For example, if one were to use XOR for $f$ as for DES, the Galois Field multiplication in AES would result in an invalid $v'$.

As such, while the general approach is applicable to all algorithms, the choice of $f$ depends on the present operations, and thus prohibits automated application of this method.
However, given multiple candidates for $f$ and a set of operators invariant for each candidate, one could perform static analysis of a program and try to find an $f$ that can tolerate all operations.
Some algorithms could then automatically be made more robust using secret sharing.
\\
\\
\dual{} related work is mostly concerned with solving the engineering challenges mentioned in \Cref{poweranalysis}.
Balancing the power consumption via $\neg{x}$ was first proposed by Saputra et. al.\cite{saputra2003masking}, where they loaded negations of memory values into a dummy register.
This reduced the effect the \hammingw{} of data had on the power consumption, and thus reduced the information leakage.
Sokolov et. al.\cite{sokolov2005design} combined the security benefits of this method with Dual-Rail encoding in general, which had previously been used in circuits without an explicit clock signal\cite{moore2002improving}.

Sokolov et. al. use the fact that both valid code words in Dual-Rail encoding have the same \hammingw{}.
By encoding every bit with this encoding they drastically reduce the effect of data on the power consumption.
\\
\\
There are multiple papers that use the powerful analysis infrastructure built around \llvm{}'s intermediate representation (\ir{}) for security.
One such paper is by Junod et. al.\cite{junod2015obfuscator}, which focuses on automatically obfuscating code during compilation.
They insert bogus instructions and replace existing instructions with equivalent ones ($a+b = a - (-b)$), all in \ir{}.
This is very similar to what my balancing pass does, albeit with a different goal.
They also change the control flow to increase the difficulty of reverse engineering, however in a constant fashion.
If the changes were dynamic and different between multiple executions, their pass might also increase robustness against \poweranalysis{}.

Lyu et. al.\cite{lyu2014dbill} also use \llvm{} and \qemu{} to analyze binaries for different platforms.
By using \qemu{} to translate machine code into \qemu{}'s intermediate representation, and then translating that into \llvm{}'s intermediate representation, they can utilize the plethora of analysis tools available for \llvm{}.
