\section{Related Work}
\label{sec:related-work}
The only paper applying defensive techniques in the compiler itself is by Junod et al.\cite{junod2015obfuscator}.
They integrate a number of software obfuscation techniques, as well as software integrity checks into \llvm{} as optimization passes.
Their approach is very close to my master thesis, however the goal is entirely different.
Junod et al. work on defending programs against reverse engineering and tampering at runtime.
Some of their passes are available online, so that can be used as a tutorial for more complex passes with more interaction with high level data structures like the syntax-tree and the control-flow-graph.

Another paper that works very closely with both \llvm{} and \qemu{} is Lyu et al.\cite{lyu2014dbill}.
They combine the \llvm{} optimizer with \qemu{} to extend static analysis tools utilizing the \llvm{} IR to different target architectures, enabling powerful desktop machines to run the analysis instead of small embedded devices.
