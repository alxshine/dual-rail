\section{Conclusion}
\label{conclusion}
In my thesis I evaluated the robustness of a compiler generated software approach to \dual{}.
By writing a proof of concept implementation I explored a new perspective on hardening embedded platforms against power analysis attacks.
The hardening is done by not using the entire available wordsize, and instead storing the inverse of data in order to balance the \hammingw{}.
Using these balanced values requires an entirely new arithmetic, which is the main contribution of this thesis.

The discussion of my work is split into three parts, for easier readability.

\subsubsection{Summary of Results}
The approach taken in my thesis works for my test code.
Balancing the binary values leads to a visible decrease in variance of \hammingw{}s.
This indicates an increased robustness against \poweranalysis{} attacks.

The arithmetic I present in this thesis is correct.
This was proven both theoretically, as well as by exhaustive search.
With balanced replacements for all operators on 8bit integers in \ir{}, my arithmetic is sound and complete.

Notably, robustness against \poweranalysis{} can be increased by not using every available bit.
This reduction in word size, along with the introduction of redundancy for balancing \hammingw{}s, drastically reduces the information an attacker can recover.
Unfortunately most binary operations cannot be executed on balanced values without imbalanced intermediates, due to the design of ALUs.
The only operator that has no imbalanced intermediates is XOR.
With my balancing choices no perfect balancing is possible for RISC architectures and current ALUs.
A different balancing scheme, using bit-interleaving, would require much more effort for logical and especially arithmetic operations, and as such would be unsuitable for a proof of concept project like my thesis.
Even then, there is no guarantee that a scheme based on interleaving performs better.

For RC4 the balancing works, although not entirely as expected.
The most frequent \hammingw{} is not where it should be, possibly caused by a very high ratio of array accesses, each requiring an unbalancing beforehand.
It is also possible that this is a general artifact of program structure, as AES shows a similar frequency for the same \hammingw{}.

For AES the balancing works very well, and the histogram looks promising.
Both its S-Box, which is copied to the stack in my code, and XOR, its most common binary operation, are perfectly balanced in my arithmetic.
This indicates a significant increase in robustness, as shown by \Cref{fig:aes}.
As such I believe my proof of concept shows the general approach is sensible, and it is reasonable to assume the increased robustness generalizes to similar cryptographic constructs.

My approach has been tested C code written for an 8bit architecture, but it should generalize to any \llvm{} compatible language whose compiler generates 8bit integers in \ir{}.
The automatic transformation means programmers can benefit from the security benefits without many considerations on their side.
As it stands, a programmer still needs to try and utilize the stack as much as possible, but this does not require expert knowledge anymore.
While only being effective for code that utilizes at most 8bit integers, this is not really a limitation for cryptographic algorithms, as these usually only operate on bytes.

With my thesis I also show that it is possible to automatically implement a generally applicable defensive measure during compilation.
I believe this highlights the great potential of modern compilers, opening multiple avenues for future work.

\subsubsection{Limitations}
Unfortunately, the performance impact of my balanced arithmetic is massive.
At least for AES, the overhead introduced by my balancing pass seems prohibitive for now.
The overhead stems from my operators being implemented via functions, which are not inlined.
This limits the amount by which code size increases, but drastically increases the amount of operations required for register storing and restoring.
Enabling inlining (and enabling compiler optimizations after my pass in general) would reduce this overhead, but for my thesis I wanted a raw evaluation of my method.

With current ALUs and a RISC architecture, an increase in the number of operations by 5 seems to be a lower bound, as that is the lowest number of intermediate operations in my balanced arithmetic.
Much larger increases are possible, as shown by balanced AES.
It is also important to note that this is for 8bit word sizes on a 32bit capable platform.
With this reduced word size it would require algorithms for handling large numbers (along the lines of GnuMP\cite{granlund1996gnu}) to reach the full 32bit value range again.
This would additionally reduce performance by a factor of at least 4.

The balancing is also incomplete, as only stack values are currently balanced.
Balancing global values (and thus all value types), would completely free the programmer from any limitations on their program.
Currently there is also no formal validation of my method.
While the histograms of \hammingw{}s suggest increased robustness, they forego any notion of time.
In order to give a more grounded security claim one would need to either attack actual hardware running balanced code, or extend \qemu{} further to capture virtual power traces during execution.
Attacking actual hardware is hard even without additional defenses in place as 32bit capable platforms run much higher clock rates than 8bit platforms, leading to much larger amounts of trace data for analysis.
Generating virtual power traces in \qemu{} would lead to the same problem of large traces, with the addition that extending \qemu{} is a massive amount of work.

\subsubsection{Future Work}
The most obvious future work for my thesis would be extending the balancing to global values.
This would reduce the considerations a programmer has to keep in mind even more, with still the same increase in robustness.

Alternatively, a nice feature would be the ability to mark certain variables for balancing, for example via special types.
This marking would then be transmitted to \ir{} by a plugin for the frontend for the current language (for C this would be Clang).
The balancing pass could then create a graph of all variables influencing the balancing target and balance all of them.
This would reduce the performance impact, while still providing increased robustness where needed.

It is currently also not clear if the balanced operations are optimal.
Future work could find either a faster arithmetic (with possible tradeoffs for robustness), or an optimality proof of the current arithmetic.
Reducing the performance impact could make this approach more generally applicable, and give it some real use cases.

In the realm of side channel defenses there are a number of possibilities for future work.
My thesis has applied a generally applicable \emph{hardware} defense automatically in software during compilation.
Following a similar approach, one could utilize static analysis in the compiler to generalize software defensive measures.
Masking could be applied automatically, with new masks (and lookup tables) being computed during compile-time, if so desired.
Secret sharing schemes could also be chosen based on the operations happening in the program.
By searching a predefined list of schemes, complete with their invariant operations, the compiler could offer applying such a scheme for increased robustness against power analysis.

Side channel defenses are only a limited subset of the possibilities of compiler enabled security in general.
The power that well defined intermediate representations in modern compilers (especially \ir{}) afford us can enable much more sophisticated security analysis than was previously possible.
Using this we should be able to automatically catch and fix many trivial security issues, and provide high level feedback where no automatic fix is possible.

Where code analysis plugins for IDEs now point out possible simplifications in logical predicates, they could point out the OWASP top ten in the future.
