
%% Begin slides template file
\documentclass[11pt,t,usepdftitle=false,aspectratio=169]{beamer}
%% ------------------------------------------------------------------
%% - aspectratio=43: Set paper aspect ratio to 4:3.
%% - aspectratio=169: Set paper aspect ratio to 16:9.
%% ------------------------------------------------------------------

\usetheme[]{uibk}
%% ------------------------------------------------------------------
%% - foot: Add a footer line for conference name and date.
%% - logo: Add the university logo in the footer (only if 'foot' set).
%% - bigfoot/sasquatch: Larger font size in footer.
%% - nototalslidenumber: Hide the total number of slides (only if 'foot' set)
%% - license: Add CC-BY license symbol to title slide (e.g., for conference uploads)
%%   (TODO: At the moment no other licenses are supported.)
%% - licenseall: Add CC-BY license symbol to all subsequent slides slides
%% - url: use \url{} rather than \href{} on the title page
%% - nosectiontitlepage: switches off the behaviour of inserting the
%%   titlepage every time a \section is called. This makes it possible to
%%   use more than one section + thanks page and a ToC off by default.
%%   If the 'nosectiontitlepage' is set you can create UIBK title slides
%%   using the command '\uibktitlepage{}' in your document to create
%%   one or multiple title slides.
%% ------------------------------------------------------------------

%% ------------------------------------------------------------------
%% The official corporate colors of the university are predefined and
%% can be used for e.g., highlighting something. Simply use
%% \color{uibkorange} or \begin{color}{uibkorange} ... \end{color}
%% Defined colors are:
%% - uibkblue, uibkbluel, uibkorange, uibkorangel, uibkgray, uibkgraym, uibkgrayl
%% The frametitle color can be easily adjusted e.g., to black with
%% \setbeamercolor{titlelike}{fg=black}
%% ------------------------------------------------------------------

%\setbeamercolor{verbcolor}{fg=uibkorange}
%% ------------------------------------------------------------------
%% Setting a highlight color for verbatim output such as from
%% the commands \pkg, \email, \file, \dataset 
%% ------------------------------------------------------------------


%% information for the title page ('short title' is the pdf-title that is shown in viewer's titlebar)
\title[Balancing binary values]{Defending against power analysis\\ by balancing binary values}
\subtitle{\large a compiler based approach}

\author[Alexander Schl\"ogl]{\small Alexander Schl\"ogl, supervised by Univ.-Prof. Dr. Rainer B\"ohme}
%('short author' is the pdf-metadata Author)
%% If multiple authors are required and the font size is too large you
%% can overrule the font size of author and url by calling:
%\setbeamerfont{author}{size*={10pt}{10pt},series=\mdseries}
%\setbeamerfont{url}{size*={10pt}{10pt},series=\mdseries}
%\URL{}
%\subtitle{}

\footertext{{\LaTeX} beamer theme}
\date{2019-09-11}

\headerimage{2}
%% ------------------------------------------------------------------
%% The theme offers four different header images based on the
%% corporate design of the university of innsbruck. Currently
%% 1, 2, 3 and 4 is allowed as input to \headerimage{...}. Default
%% or fallback is '1'.
%% ------------------------------------------------------------------

\begin{document}

%% ALTERNATIVE TITLEPAGE
%% The next block is how you add a titlepage with the 'nosectiontitlepage' option, which switches off
%% the default behavior of creating a titlepage every time a \section{} is defined.
%% Then you can use \section{} as it's originally intended, including a table of contents.
% \usebackgroundtemplate{\includegraphics[width=\paperwidth,height=\paperheight]{titlebackground.pdf}}
% \begin{frame}[plain]
%     \titlepage
% \end{frame}
% \addtocounter{framenumber}{-1}
% \usebackgroundtemplate{}

%% Table of Contents, if wanted:
%% this requires the 'nosectiontitlepage' option and setting \section{}'s as you want them to appear here.
%% Subsections and subordinates are suppressed in the .sty at the moment, search
%% for \setbeamertemplate{subsection} and replace the empty {} with whatever you want.
%% Although it's probably too much for a presentation, maybe for a lecture.
%% Please note: \maketitle allows you to render a uibk-style title page wherever needed
%% in the document even if 'nosectiontitlepage' option is set (note: \maketitle will not
%% create a new section and is therefore not included in \tableofcontents (if used).
% \maketitle
% \begin{frame}
%     \vspace*{1cm plus 1fil}
%     \tableofcontents
%     \vspace*{0cm plus 1fil}
% \end{frame}


%% this sets the first PDF bookmark and triggers generation of the title page
\section{Bookmark Title}

%% this just generates PDF bookmarks
\subsection{Overview}

%% first slide
\begin{frame}

  \frametitle{Overview}
  \textbf{Content}
  
  \begin{itemize}
  \item Power analysis
  \item Approach
  \item Selected Details
  \item Results
  \item Future Work
  \end{itemize}

\end{frame}

\subsection{Power Analysis}

\begin{frame}

\end{frame}
\begin{frame}

\end{frame}
\begin{frame}

\end{frame}
\begin{frame}

\end{frame}
\begin{frame}

\end{frame}

\subsection{Approach}
\begin{frame}

\end{frame}
\begin{frame}

\end{frame}

\subsection{Selected Details}
\begin{frame}

\end{frame}
\begin{frame}

\end{frame}
\begin{frame}

\end{frame}

\subsection{Compiler Pass}
\begin{frame}

\end{frame}
\begin{frame}

\end{frame}
\begin{frame}

\end{frame}
\begin{frame}

\end{frame}

\subsection{Evaluation}
\begin{frame}

\end{frame}
\subsection{Results}
\begin{frame}

\end{frame}
\begin{frame}

\end{frame}
\begin{frame}

\end{frame}
\begin{frame}

\end{frame}
\begin{frame}

\end{frame}
\begin{frame}

\end{frame}
\begin{frame}

\end{frame}

\subsection{Future Work}
\begin{frame}

\end{frame}

\subsection{Conclusion}
\begin{frame}

\end{frame}

\end{document}
