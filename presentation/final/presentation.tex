
%% Begin slides template file
\documentclass[11pt,t,usepdftitle=false,aspectratio=169]{beamer}
%% ------------------------------------------------------------------
%% - aspectratio=43: Set paper aspect ratio to 4:3.
%% - aspectratio=169: Set paper aspect ratio to 16:9.
%% ------------------------------------------------------------------

% \usepackage{enumitem}

\usetheme[nototalframenumber]{uibk}
%% ------------------------------------------------------------------
%% - foot: Add a footer line for conference name and date.
%% - logo: Add the university logo in the footer (only if 'foot' set).
%% - bigfoot/sasquatch: Larger font size in footer.
%% - nototalslidenumber: Hide the total number of slides (only if 'foot' set)
%% - license: Add CC-BY license symbol to title slide (e.g., for conference uploads)
%%   (TODO: At the moment no other licenses are supported.)
%% - licenseall: Add CC-BY license symbol to all subsequent slides slides
%% - url: use \url{} rather than \href{} on the title page
%% - nosectiontitlepage: switches off the behaviour of inserting the
%%   titlepage every time a \section is called. This makes it possible to
%%   use more than one section + thanks page and a ToC off by default.
%%   If the 'nosectiontitlepage' is set you can create UIBK title slides
%%   using the command '\uibktitlepage{}' in your document to create
%%   one or multiple title slides.
%% ------------------------------------------------------------------

%% ------------------------------------------------------------------
%% The official corporate colors of the university are predefined and
%% can be used for e.g., highlighting something. Simply use
%% \color{uibkorange} or \begin{color}{uibkorange} ... \end{color}
%% Defined colors are:
%% - uibkblue, uibkbluel, uibkorange, uibkorangel, uibkgray, uibkgraym, uibkgrayl
%% The frametitle color can be easily adjusted e.g., to black with
%% \setbeamercolor{titlelike}{fg=black}
%% ------------------------------------------------------------------

%\setbeamercolor{verbcolor}{fg=uibkorange}
%% ------------------------------------------------------------------
%% Setting a highlight color for verbatim output such as from
%% the commands \pkg, \email, \file, \dataset 
%% ------------------------------------------------------------------

\usepackage{tikz}
\usetikzlibrary{arrows,decorations.pathreplacing}
\tikzset{>=stealth}

% argument #1: any options
\newenvironment{customlegend}[1][]{%
    \begingroup
    % inits/clears the lists (which might be populated from previous
    % axes):
    \csname pgfplots@init@cleared@structures\endcsname
    \pgfplotsset{#1}%
}{%
    % draws the legend:
    \csname pgfplots@createlegend\endcsname
    \endgroup
}%

% makes \addlegendimage available (typically only available within an
% axis environment):
\def\addlegendimage{\csname pgfplots@addlegendimage\endcsname}

\tikzset{candidate/.style={draw, align=center, rounded corners=0.1cm, fill=white}}
\tikzset{balanced/.append style={draw=uibkorange, thick, fill=uibkorange!40}}
\tikzset{todo/.append style={draw=uibkorange, thick, pattern=north west lines, pattern color=uibkorang!40}}
\tikzset{target/.style={align=center, draw, thick, fill=white}}

\usepackage{adjustbox}
\usepackage{bm}
\usepackage{amsmath}
\usepackage{listings}
\lstset{escapechar=`}

\usepackage[norndcorners,customcolors]{hf-tikz}
\hfsetbordercolor{uibkorange}
\hfsetfillcolor{uibkorangel}

%% information for the title page ('short title' is the pdf-title that is shown in viewer's titlebar)
\title[Balancing binary values]{Defending against power analysis\\ by balancing binary values}
\subtitle{\large a compiler based approach}

\author[Alexander Schl\"ogl]{\small Alexander Schl\"ogl, supervised by Univ.-Prof. Dr. Rainer B\"ohme}
%('short author' is the pdf-metadata Author)
%% If multiple authors are required and the font size is too large you
%% can overrule the font size of author and url by calling:
%\setbeamerfont{author}{size*={10pt}{10pt},series=\mdseries}
%\setbeamerfont{url}{size*={10pt}{10pt},series=\mdseries}
%\URL{}
%\subtitle{}

\date{2019-09-11}

\headerimage{2}
%% ------------------------------------------------------------------
%% The theme offers four different header images based on the
%% corporate design of the university of innsbruck. Currently
%% 1, 2, 3 and 4 is allowed as input to \headerimage{...}. Default
%% or fallback is '1'.
%% ------------------------------------------------------------------

\usepackage{graphicx}
\graphicspath{ {fig/}}

\newcommand\blfootnote[1]{%
  \begingroup
  \renewcommand\thefootnote{}\footnote{#1}%
  \addtocounter{footnote}{-1}%
  \endgroup
}
\hyphenation{consumption}
\hyphenation{algorithm}

\newcommand{\hex}[1]{\texttt{0x#1}}

\renewcommand{\neg}[1]{\ensuremath{\overline{#1}}}
\newcommand{\bsep}{\; \| \; }
\newcommand{\borr}{\mathbin{\texttt{ORR}}}
\newcommand{\band}{\mathbin{\texttt{AND}}}
\newcommand{\bxor}{\mathbin{\texttt{XOR}}}
\newcommand{\bror}{\mathbin{\texttt{ROR}}}
\newcommand{\blsl}{\mathbin{\texttt{LSL}}}
\newcommand{\trans}[2]{\ensuremath{\texttt{transform\_#1\_#2}}}

\newcommand{\binp}[5]{\ensuremath{\%#1 &= #2 &&\bsep #3 &&\bsep #4 &&\bsep #5 &&}}
\newcommand{\btrans}[6]{\ensuremath{\%#1 &= #2 &&\bsep #3 &&\bsep #4 &&\bsep #5 && \;|\ #6}}

\newcommand{\vq}{\vphantom{q}}

\begin{document}

%% ALTERNATIVE TITLEPAGE
%% The next block is how you add a titlepage with the 'nosectiontitlepage' option, which switches off
%% the default behavior of creating a titlepage every time a \section{} is defined.
%% Then you can use \section{} as it's originally intended, including a table of contents.
% \usebackgroundtemplate{\includegraphics[width=\paperwidth,height=\paperheight]{titlebackground.pdf}}
% \begin{frame}[plain]
%     \titlepage
% \end{frame}
% \addtocounter{framenumber}{-1}
% \usebackgroundtemplate{}

%% Table of Contents, if wanted:
%% this requires the 'nosectiontitlepage' option and setting \section{}'s as you want them to appear here.
%% Subsections and subordinates are suppressed in the .sty at the moment, search
%% for \setbeamertemplate{subsection} and replace the empty {} with whatever you want.
%% Although it's probably too much for a presentation, maybe for a lecture.
%% Please note: \maketitle allows you to render a uibk-style title page wherever needed
%% in the document even if 'nosectiontitlepage' option is set (note: \maketitle will not
%% create a new section and is therefore not included in \tableofcontents (if used).
% \maketitle
% \begin{frame}
%     \vspace*{1cm plus 1fil}
%     \tableofcontents
%     \vspace*{0cm plus 1fil}
% \end{frame}


%% this sets the first PDF bookmark and triggers generation of the title page
\section{Motivation}
\begin{frame}
  \frametitle{Motivation}
  \begin{figure}
    \begin{columns}[T]
      \begin{column}{0.48\textwidth}
        \centering
        \vfill
        \includegraphics[width=0.8\textwidth]{hue.jpeg}
        \vfill
      \end{column}
      \hfill
      \begin{column}{0.48\textwidth}
        \centering
        \vspace{1cm}
        \includegraphics[width=0.7\textwidth]{bank-card.jpg}
        \vfill
      \end{column}
    \end{columns}
  \end{figure}
  \blfootnote{\tiny
https://store.storeimages.cdn-apple.com/4982/as-images.apple.com/is/HJCC2?wid=1144\&hei=1144\&fmt=jpeg\&qlt=95\&op\_usm=0.5,0.5\&.v=0}
  \blfootnote{\tiny
    https://www.liberaldictionary.com/wp-content/uploads/2019/01/bank-card-8123.jpg}
\end{frame}

\begin{frame}
  \frametitle{Motivation}
  \begin{figure}
    \centering
    \begin{tikzpicture}[every node/.style={draw}]
      \node (input) at (-3, 2) {\vq \only<1>{Input}\only<2>{\textcolor{uibkorange}{Known Input}}\only<3->{Known Input}};
      \node (secret) at (-3, -2) {\vq Secret};
      \node (result) at (-0.5,0) {\vq Result};
      \node[text width=2.5cm, align=center] (power) at (4,0) {\vq Power Consumption};

      \draw[->] (input) -- (result);
      \draw[->] (secret) -- (result);
      \draw[->] (result) -- (power);

      \onslide<2>{
        % \node[uibkorange,thick] (known) at (-1.5, 2) {\vq known};
        % \draw[->, uibkorange, line width=0.4mm] (known) -- (input);
        \draw[->, uibkorange, line width=0.4mm] (6,-3.5) -- (-4.5, -3.5) node[midway, above, draw=none] {Power Analysis Attack};
      }
      \onslide<3>{
        % \node[uibkblue, thick] (knownb) at (-1.5, 2) {\vq known};
        % \draw[->, uibkblue, line width=0.4mm] (knownb) -- (input);
        
        \draw[->, uibkblue, line width=0.4mm] (6,-3.5) -- (-4.5, -3.5) node[midway, above, draw=none] {Power Analysis Attack};
        \node[uibkorange, thick] (mask) at (-3, 0) {\vq Mask};
        \draw[uibkorange, thick, ->] (mask) -- (result);

        \draw[uibkorange, line width=0.4mm] (1.45,0.15) -- (1.15,-0.15);
        \draw[uibkorange, line width=0.4mm] (1.15,0.15) -- (1.45,-0.15);

        \draw[uibkorange, thick,<-] (1.3,-0.3) -- (1.3,-1) node[below, draw=none] {Dual-Rail Logic};
      }
    \end{tikzpicture}
  \end{figure}
\end{frame}

\begin{frame}
  \frametitle{Motivation}
  \vfill
  \begin{columns}[T]
    \begin{column}{0.48\textwidth}
      \begin{block}{Masking}
        Increases analysis complexity
        \begin{itemize}
        \item[+] Runs on standard hardware
        \item[--\hspace{0.4mm}] Built into algorithm
        \item[--\hspace{0.4mm}] Requires expert knowledge
        \end{itemize}
      \end{block}
    \end{column}
    %% \hfill
    \begin{column}{0.48\textwidth}
      \begin{block}{Dual-Rail Logic}
        Balances power consumption
        \begin{itemize}
        \item[+] Can run any program
        \item[--\hspace{0.4mm}] Requires specialized hardware
        \item[] \vq
        \end{itemize}
      \end{block}
    \end{column}
  \end{columns}
  \vfill
  \onslide<2>{
    \begin{alertblock}{Best of both worlds?}
      Apply balancing similar to Dual-Rail logic in software
    \end{alertblock}
  }
  \vfill
\end{frame}

%% this just generates PDF bookmarks
% \section{Overview}

%% first slide
\begin{frame}<1>[label=overview]
  \frametitle{Overview}
  \vfill
  \textbf{Content}

  \begin{itemize}
  \item Motivation
  \item \textcolor<2>{uibkorange}{Balancing}
  \item \textcolor<3>{uibkorange}{Arithmetic}
  \item \textcolor<4>{uibkorange}{Code Transformation}
  \item \textcolor<5>{uibkorange}{Results}
  \item \textcolor<6>{uibkorange}{Conclusion}
  \end{itemize}
  \vfill
\end{frame}

% \section{Balancing}
\begin{frame}
  \frametitle{Balancing}
  \vfill
  \textbf{Working assumption:}\\
  Power consumption is proportional to Hamming weight (number of 1s)\\
  $\rightarrow$ constant Hamming weight = constant power consumption
  \onslide<2->{
  \begin{alertblock}{Approach}
    Extend register size, and store inverse along with actual value
  \end{alertblock}
  \vspace{0.3cm}
  \begin{center}
    \begin{tikzpicture}
      \foreach \i in {0,8}{
        \draw (8-\i/4,0.2) -- (8-\i/4,-0.2);
        \node at (8-\i/4,-0.5) {\tiny \i};
      }
      \node at (7, 0) {\Large\texttt{x}};         

      \onslide<3>{
      \foreach \i in {16,24,32}{
        \draw (8-\i/4,0.2) -- (8-\i/4,-0.2);
        \node at (8-\i/4,-0.5) {\tiny \i};
      }
      \node at (5, 0) {\Large\texttt{0}};
      \node[uibkorange] at (3, 0) {\Large\neg{\texttt{x}}};
      \node at (1, 0) {\Large\texttt{0}};
      }
    \end{tikzpicture}
  \end{center}
  }
\end{frame}

% \againframe<3>{overview}

% \section{Arithmetic}
\begin{frame}
  \frametitle{Arithmetic}
  Regular operators will not work:
  \begin{center}
    \begin{tikzpicture}
      \foreach \y in {2,0,-2}{
        \foreach \i in {0,8,...,32}{
          \draw (8-\i/4,\y+0.2) -- (8-\i/4,\y-0.2);
          \node at (8-\i/4,\y-0.5) {\tiny \i};
        }
        \node at (5, \y) {\Large\texttt{0}};
        \node at (1, \y) {\Large\texttt{0}};
      }
      \node at (7, 2) {\Large\texttt{x}};
      \node at (3, 2) {\Large\neg{\texttt{x}}};

      \node at (4,.8) {\Large{$\bm{\lor}$}};

      \node at (7, 0) {\Large\texttt{y}};
      \node at (3, 0) {\Large\neg{\texttt{y}}};

      \node at (4,-1.2) {\Large{$\bm{=}$}};

      \node at (7, -2) {\Large$\texttt{x} \bm{\lor} \texttt{y}$};
      \node at (3, -2) {\Large$\neg{\texttt{x}} \bm{\lor} \neg{\texttt{y}}$};

      \node[uibkorange] at (3, -2.6) {\Large$\bm{\neq}$};
      \node at (3.91, -3.2) {\Large$\neg{\texttt{x} \bm{\lor} \texttt{y}} = \neg{\texttt{x}} \bm{\land} \neg{\texttt{y}}$};
  \end{tikzpicture}
  \end{center}
\end{frame}

\begin{frame}[fragile]
  \frametitle{Arithmetic}

  \begin{columns}[T] % align columns
    \begin{column}{.28\textwidth}
      Find replacements for:
      \begin{itemize}
      \item \textcolor<2>{uibkorange}{\texttt{ORR}} \only<2>{\quad\tikz[baseline=0.5ex]{\draw[->,uibkorange,line width=0.3mm] (0,0.2) -- (2,0.2);}}
      \item \texttt{AND}
      \item \texttt{XOR}
      \item \texttt{ADD}
      \item \texttt{SUB}
      \item \texttt{MUL}
      \item \texttt{SHIFTS}
      \item \texttt{DIV}
      \item \texttt{REM}
      \end{itemize}
    \end{column}%
    \hfill%
    \pause
    \vrule
    \hfill
    \begin{column}{.6\textwidth}
      \vspace{1cm}
      \begin{lstlisting}[language=C, basicstyle=\small]
int balanced_or(int lhs,
        int rhs) {
  int temp_or = lhs | rhs;
  int temp_and = lhs & rhs;
  int combined = (temp_and << 8)
          | temp_or;
  combined &= 0xff0000ff;
  return balanced_2_1(combined);
}
      \end{lstlisting}
    \end{column}
  \end{columns}
\end{frame}

\begin{frame}
  \frametitle{Verifying the arithmetic}
  Perform exhaustive search of the input space:
  \begin{columns}[T]
    \begin{column}{0.48\textwidth}
      \begin{block}{Test framework}
        \begin{itemize}
        \item Takes individual steps as lambdas
        \item Executes over all inputs
        \item Stores intermediate values
        \item \textcolor<2>{uibkorange}{Checks correctness}
        \item \textcolor<3>{uibkorange}{Plots Hamming weight histograms}
        \end{itemize}        
      \end{block}
    \end{column}
    \hfill
    \onslide<3>{
    \begin{column}{0.48\textwidth}
      \centering
      \includegraphics[width=\textwidth]{orr.png}
    \end{column}
    }
  \end{columns}
\end{frame}

% \againframe<4>{overview}

% \section{Code Transformation}
\begin{frame}
  \frametitle{Applying the changes}
  \begin{alertblock}{Automatic balancing}
    Rewrite code during compilation
  \end{alertblock}

  \vfill
  \onslide<2->{
  LLVM:
  \begin{figure}
  \centering
  \begin{tikzpicture}
    \node[draw, minimum width=1.3cm, minimum height=0.6cm] at (-2.5,1.5) (C) {C};
    \node[draw, minimum width=1.3cm, minimum height=0.6cm] at (-2.5,0.5) (C++) {C++};
    \node[draw, minimum width=1.3cm, minimum height=0.6cm] at (-2.5, -0.5) (Haskell) {Haskell};
    \node[draw, minimum width=1.3cm, minimum height=0.6cm] at (-2.5, -1.5) (otherl) {...};
    \node[draw] at (-0.75, 0) (irl) {IR};

    \draw[decoration={brace,mirror,amplitude=6pt}, decorate, line width=0.3mm, uibkgray!70] (-3.3,-1.9) -- (-1.7, -1.9) node[midway, below, yshift=-3pt] {\scriptsize Frontends};
    
    \node[draw, label=Optimizer Passes, minimum width=4cm, minimum height=1.3cm] at (2,0) (optimization) {};
    \node[draw, minimum size=0.5cm] at (0.6,0) {};
    \node[draw, minimum size=0.5cm] at (1.2,0) {};
    \node[draw, minimum size=0.5cm] at (1.8,0) {};
    \node[draw, minimum size=0.5cm] at (2.4,0) {};
    \node at (3.0, 0) {...};

    \node[draw] at (4.75, 0) (irr) {IR};
    \node[draw, minimum width=1.3cm, minimum height=0.6cm] at (6.5,1) (x86) {x86};
    \node[draw, minimum width=1.3cm, minimum height=0.6cm] at (6.5,0) (arm) {ARM};
    \node[draw, minimum width=1.3cm, minimum height=0.6cm] at (6.5,-1) (otherr) {...};

    \draw[decoration={brace,mirror,amplitude=6pt}, decorate, line width=0.3mm, uibkgray!70] (5.7,-1.9) -- (7.3, -1.9) node[midway, below, yshift=-3pt] {\scriptsize Backends};
    
    \draw[->] (C) -- (irl);
    \draw[->] (C++) -- (irl);
    \draw[->] (Haskell) -- (irl);
    \draw[->] (otherl) -- (irl);

    \draw[->] (irl) -- (optimization);
    \draw[->] (optimization) -- (irr);

    \draw[->] (irr) -- (x86);
    \draw[->] (irr) -- (arm);
    \draw[->] (irr) -- (otherr);

    \onslide<3>{
    \node[draw, minimum height=0.6cm] (balance) at (3,-2) {Balance.cpp};
    \draw[->, uibkorange, line width=0.3mm, shorten <= 5pt] (balance) -- (3,-0.2);
    }
  \end{tikzpicture}
  \end{figure}
  }
\end{frame}

\begin{frame}
  \frametitle{Optimizer Pass}
  \centering
  \begin{tikzpicture}[scale=0.6, every node/.style={transform shape}]
    \draw[fill=uibkblue!10] (1.7, 0.4) ellipse (7cm and 5.7cm);
    \node[align=center] at (-2.8, -2.1) {HEAP / GLOBALS};

    \draw[fill=uibkorange!10] (1.2,2) ellipse (5.7cm and 3.7cm);
    \node at (-1.7, 0) {STACK / LOCALS};

    \node at (-4.5, -4.2) {BUSSES};

    \node (literals) [candidate] at (-1.7,3.7) {literals};
    \node (ops) [candidate] at (-3,1.5) {binary \\ operations};
    \node (lvars) [candidate] at (0.8,2) {local \\ variables};
    \node (larrs) [candidate] at (1.7,-0.5) {local \\ arrays};
    \node (params) [candidate] at (3.5, 4) {function \\ parameters};
    \node (returns) [candidate] at (4, 2.5) {return \\ values};
    \node (registers) [target] at (4.3, 0) {registers};

    \draw[->] (literals) -> (lvars);
    \draw[->] (ops) -> (lvars);
    \draw[->] (lvars) -> (larrs);
    \draw[<->] (lvars) -> (params);
    \draw[<->] (lvars) -> (returns);

    \draw[->] (lvars) -> (registers);
    \draw[->] (larrs) -> (registers);

    \node (gvars) [candidate] at (0.5, -2.7) {global \\variables};
    \node (garrs) [candidate] at (3.2, -2.6) {global \\arrays};
    \node (constants) [candidate] at (0.8, -3.9) {constants};
    \node (memory) [target] at (3.4, -4.2) {main \\memory};
    \node (pointers) [candidate] at (6.9, -1.0) {pointers};

    \draw[->] (lvars) -> (gvars);
    \draw[->] (larrs) -> (garrs);
    \draw[->] (constants) -> (gvars);
    \draw[->] (gvars) -> (memory);
    \draw[->] (garrs) -> (memory);

    \node (databus) [target] at (6.7, -4.7) {data \\ bus};
    \node (addressbus) [target] at (8.9, -3.0) {address \\ bus};

    \draw[->] (registers) -> (databus);
    \draw[->] (memory) -> (databus);
    \draw[->] (pointers) -> (addressbus);
  \end{tikzpicture}
\end{frame}

\begin{frame}[label=implemented]
  \frametitle{Optimizer Pass}
  \centering
  \begin{tikzpicture}[scale=0.6, every node/.style={transform shape}]
    \draw[fill=uibkblue!10] (1.7, 0.4) ellipse (7cm and 5.7cm);
    \node[align=center] at (-2.8, -2.1) {HEAP / GLOBALS};

    \draw[fill=uibkorange!10] (1.2,2) ellipse (5.7cm and 3.7cm);
    \node at (-1.7, 0) {STACK / LOCALS};

    \node at (-4.5, -4.2) {BUSSES};

    \node (literals) [candidate, balanced] at (-1.7,3.7) {literals};
    \node (ops) [candidate, balanced] at (-3,1.5) {binary \\ operations};
    \node (lvars) [candidate, balanced] at (0.8,2) {local \\ variables};
    \node (larrs) [candidate, balanced] at (1.7,-0.5) {local \\ arrays};
    \node (params) [candidate, balanced] at (3.5, 4) {function \\ parameters};
    \node (returns) [candidate, balanced] at (4, 2.5) {return \\ values};
    \node (registers) [target, balanced] at (4.3, 0) {registers};

    \draw[->] (literals) -> (lvars);
    \draw[->] (ops) -> (lvars);
    \draw[->] (lvars) -> (larrs);
    \draw[<->] (lvars) -> (params);
    \draw[<->] (lvars) -> (returns);

    \draw[->] (lvars) -> (registers);
    \draw[->] (larrs) -> (registers);

    \node (gvars) [candidate] at (0.5, -2.7) {global \\variables};
    \node (garrs) [candidate] at (3.2, -2.6) {global \\arrays};
    \node (constants) [candidate] at (0.8, -3.9) {constants};
    \node (memory) [target] at (3.4, -4.2) {main \\memory};
    \node (pointers) [candidate] at (6.9, -1.0) {pointers};

    \draw[->] (lvars) -> (gvars);
    \draw[->] (larrs) -> (garrs);
    \draw[->] (constants) -> (gvars);
    \draw[->] (gvars) -> (memory);
    \draw[->] (garrs) -> (memory);

    \node (databus) [target] at (6.7, -4.7) {data \\ bus};
    \node (addressbus) [target] at (8.9, -3.0) {address \\ bus};

    \draw[->] (registers) -> (databus);
    \draw[->] (memory) -> (databus);
    \draw[->] (pointers) -> (addressbus);
  \end{tikzpicture}
\end{frame}

\begin{frame}
  \frametitle{Optimizer Pass}

  \vfill
  Additionally transform:
  \begin{itemize}
  \item stores
  \item loads
  \item casts
  \item array indexing
  \item compares
  \item function calls
  \end{itemize}
  \vfill
\end{frame}

% \section{Results}
\begin{frame}
  \frametitle{Evaluation}

  How to generate ``virtual'' power traces?
  
  \begin{block}{Qemu alone}
    \begin{itemize}
    \item[+] fast
    \item[--\hspace{0.4mm}] incorrect view
    \end{itemize}
  \end{block}
  \pause
  \begin{alertblock}{Qemu + gdb}
    \begin{itemize}
    \item[+] correct view
    \item[+] includes program location information
    \item[--\hspace{0.4mm}] \textbf{very} slow
    \end{itemize}
    Execute instruction by instruction, dump registers every time
  \end{alertblock}
\end{frame}

\begin{frame}[label=results]
  \frametitle{Results}
  \begin{alertblock}{No attack}
    No attack was mounted, instead performed statistical analysis
  \end{alertblock}
  \center
  \onslide<2->{
  \vfill
  \begin{tabular}{|l|l|l|}
    \hline
    & \multicolumn{2}{c|}{AES} \\
    \cline{2-3}
    & unbalanced & balanced \\
    \cline{2-3}
    Executed instructions & 22 876 & 339 168 \\
    Relative increase & 1 & 14.888 \\
    Balanced operations & 20 571 & 334 521 \\
    Balancedness      & 0.903 & 0.986 \\
    Unbalanced operations & 2211 & 4647 \\
    \hline
  \end{tabular}
  }
  \vfill
\end{frame}

% \againframe<4>{results}

% \section{Conclusion}
% \againframe<6>{overview}

\begin{frame}
  \frametitle{Summary}
  \vfill
  \begin{itemize}
  \item Arithmetic is \emph{mostly} proven to be correct
  \item Works without programmer work
  \item Balances everything on stack
  \item Requires more powerful, but standard hardware
  \item Does not explode code size
  \end{itemize}
  \vfill
\end{frame}

\begin{frame}
  \frametitle{Limitations}
  \vfill
  \begin{itemize}
  \item Works only on stack
  \item Only tested on some code samples
  \item Correctness of \texttt{REM} and \texttt{DIV} not proven
  \item Not attacked, only evaluated
  \item Greatly increased execution time
  \end{itemize}
  \vfill
\end{frame}

\begin{frame}
  \frametitle{Future work}
  Improve on thesis:
  \begin{itemize}
  \item Test on actual hardware
  \item Balance globals
  \item Improve operators
  \item Mark balancing targets
  \end{itemize}
  \vspace{0.5cm}
  Similar ideas:
  \begin{itemize}
  \item Move balancing to type system
  \item Other power analysis defenses
  \item Control flow randomization
  \item Move more security tools to LLVM
  \end{itemize}
\end{frame}

\begin{frame}
  \frametitle{Conclusion}
  \begin{itemize}
  \item Debugging optimizer passes is hard
  \item Security and performance likely mutually exclusive
  \item Backend cannot entirely be ignored
  \item Qemu is not a processor emulator
  \end{itemize}
  \vfill
  \begin{block}{LLVM IR}
    LLVM's intermediate representation offers many avenues for future work,\\
    not only for optimizition, but also for security.
  \end{block}
\end{frame}

\begin{frame}
\end{frame}

\appendix

\begin{frame}
  \begin{figure}
    \centering
    \includegraphics[height=\textheight]{aes-parts.png}
  \end{figure}
\end{frame}

\begin{frame}
  \begin{figure}
    \centering
    \includegraphics[height=\textheight]{imbalances-0.png}
  \end{figure}
\end{frame}

\begin{frame}
  \begin{figure}
    \centering
    \includegraphics[height=\textheight]{imbalances-1.png}
  \end{figure}

\end{frame}
\begin{frame}
  \begin{figure}
    \centering
    \includegraphics[height=\textheight]{imbalances-2.png}
  \end{figure}

\end{frame}
\begin{frame}
  \begin{figure}
    \centering
    \includegraphics[height=\textheight]{imbalances-3.png}
  \end{figure}
\end{frame}

\begin{frame}[label=results]
  \frametitle{Results}
  \begin{alertblock}{No attack}
    No attack was mounted, instead performed statistical analysis
  \end{alertblock}
  \center
  \vfill
  \begin{tabular}{|l|c|c|c|}
    \hline
    & \multicolumn{3}{c|}{AES} \\
    \cline{2-4}
    & unbalanced & balanced & filtered balanced \\
    \cline{2-4}
    Executed instructions & 22 876 & 339 168 & 339 168\\
    Relative increase & 1 & 14.888 & 14.888 \\
    Balanced operations & 20 571 & 334 521 & 337 852\\
    Unbalanced operations & 2211 & 4647 & 1316 \\
    Balancedness      & 0.903 & 0.986 & 0.996\\
    \hline
  \end{tabular}
  \vfill
\end{frame}

\chapter{Power Analysis}
\label{poweranalysis}
Power analysis utilizes the fact that different operations have different power consumptions.
By capturing the power consumption traces (power traces in short) and examining them, an attacker can reason about the variables determining the control flow during program execution.
If these variables are cryptographic secrets, or keys (e.g. private keys in RSA), they are leaked to the attacker in what is called a simple power analysis (SPA) attack\cite{kocher1999differential}.
In many cases the control is not related to the key, requiring more complex attacks.
For these cases capturing a large number of traces and performing statistical analysis can leak information about \emph{individual values}.
The two main attacks of this type are called differential power analysis (DPA)\cite{kocher1999differential} and correlation power analysis (CPA)\cite{brier2004correlation}.
All variants and the setup required are explained in this section.

\section{Setup}
Performing \poweranalysis{} requires access to the device in a controlled environment.
The attacker needs to control the power supply for the target, and be able to alter the containing electronics.
This is fairly easy for embedded platforms, as they are often made to be self-contained.
Even for more complex targets like IoT devices, the processor can be removed from its circuit board and put in an attack environment\cite{ronen2017iot}.

The setup for a \poweranalysis{} attack is as follows:
An attacker solders a resistor between the target processor and the ground connector of its power supply.
She then measures the voltage difference between both ends of the resistor with an oscilloscope.
This voltage is directly proportional to the current flowing through the resistor, and thus to the power consumption of the target.
The voltage recordings from the oscilloscope are then transferred to a computer, and can be analyzed there.

\section{Simple Power Analysis}
In the simplest form of \poweranalysis{}, SPA, power traces are usually examined visually.
Repeating patterns of operations can often be identified, leaking information about the control flow.
If the control flow is dependent on a key, an attacker can infer its value.
An example target would be RSA decryption being calculated via the square-and-multiply algorithm.
The difference between the multiply and the square operation is directly observable from power traces.
As the order of these operations is linked to the private key, identifying the control flow leaks the private key.
\Cref{fig:spa} shows a trace for square-and-multiply in RSA decryption, including the leaked private key bits.

\begin{figure}[h]
  \centering
  \includegraphics[width=\textwidth]{spa.png}
  \caption{Simple power analysis on square-and-multiply RSA\cite{boehme2017netsec}}
  \label{fig:spa}
\end{figure}

Power traces are often not as clear as in \Cref{fig:spa}, and contain some amount of random noise.
By averaging multiple traces this noise can be reduced, allowing for easier analysis.

\section{Differential Power Analysis}
The control flow is often not enough to leak the entire key, and it is very hard to gain information about the actual data from only SPA.
For this a more complex variant of \poweranalysis{} can be used, namely \emph{Differential \poweranalysis{}} (DPA).
DPA requires a large number of power traces, with the cryptographic algorithm being run on a known plaintext in each execution.
It also requires that the attacker knows which cryptographic algorithm is running on the target.
Both of these requirements are usually fulfilled in embedded devices.

The attacker additionally requires a \emph{power model}, which is used to calculate the expected power consumption for a given value.
While not entirely accurate\cite{brier2004correlation}, the \hammingw{} has proven to be a very effective power model in practice.

Knowing the algorithm running on the target, the attacker can predict the result of intermediate computations, given some assumption about the key.
In practice she usually guesses a single key bit, and then calculates the expected result of some bitwise operation, e.g. XOR.
The other bits are not important for her attack at the moment and are thus ignored.
After calculating the expected power consumption for all plaintexts, she splits the power traces in two subsets, based on the value of the expected result.
Then she calculates the mean of both sets, and calculates their difference.
If the guess for the current key bit was correct, then all traces in a single subset will have the same value for the current bit.
This means that the difference between both means shows spikes in the places where the attacked operation happens.

If the guess was wrong on the other hand, the values of the current bit are randomly distributed in both sets, and the difference will show no spikes of non-zero values.
The values of the other bits can be ignored, as they are close enough to being random in both sets, and thus filtered out by calculating the difference.
Other power consumption noise, coming from different parts of the processor, is randomly distributed overy every trace, and thus filtered by taking the mean.
Both the values for other bits and other power consumption influences are note entirely random, but close enough to it that these assumptions work well in practice.

\Cref{fig:dpa} shows a typical DPA result with the mean power consumptions of both sets, the difference between the two, and the difference with the Y axis magnified by a factor of 15.
This analysis was performed on the output of the least significant bit after the first S-box substitution in AES.
\begin{figure}[h]
  \centering
  \includegraphics[width=0.8\textwidth]{dpa.png}
  \caption{Difference in means during DPA\cite{kocher2011introduction}}
  \label{fig:dpa}
\end{figure}

\section{Correlation Power Analysis}
CPA is the most complex of these attacks, but also offers the best results.
The attacker starts by making a list of candidate values for a part of the key, usually individual bytes.
With a power model, usually the \hammingw{} like for DPA, she can calculate the expected power consumption for combination of key guess with every plaintext.

For a fixed key guess, these expected power consumptions are a function over all plaintexts.
For a single point in time, the actual power traces are \emph{also} a function over all plaintexts, with the actual key being a fixed parameter unknown to the attacker.
By testing how much the expected power consumptions correlate with the actual power consumptions, the attacker can find the confidence values of the guesses for the current part of the key.
She then assumes the value with the highest correlation coefficient is correct, and continues her attack for the rest of the key.

Of course the attacker does not know exactly which point in time she is supposed to examine.
However, it is very unlikely that an incorrect key guess at a random point in time has a closer correlation than the correct key guess at the correct time, and so an attacker can simply use the maximum correlation coefficient.

\section{Defenses Against Power Analysis Attacks}
Defenses against this class of attack usually work by either adding additional factors to the power consumption, thus increasing the computational effort required for analysis, or by reducing variances in power consumption altogether.
This reduced variance decreases the information an attacker can gain from the same number of power traces, giving her reduced confidence in her result or requiring her to capture more traces.

Masking\cite{golic2002multiplicative}\cite{coron2000boolean} for example is an algorithm specific defensive measure that adds a third factor to the power consumption by first performing adding a masking value to the plaintext via an invertible operation.
The cryptographic algorithm then works on the masked value, and only in the end unmasks the result.
As such the attacker has to calculate her correlation for each possible combination of key byte and mask value.
This increases the number of traces she needs to capture (to still provide the same confidence in her analysis) and the computation time of her analysis.

Other defensive measures focus on creating a worse signal to noise ratio for the entire power consumption.
One technique that has gained a lot of traction is \dual{}\cite{sokolov2005design}.
It works by calculating the inverse of every intermediate result along with the actual result, thus balancing the number of 1s.
This results in a constant \hammingw{} and therefore a data-independent power consumption.

Unfortunately, \dual{} suffers from multiple engineering problems.
The power required to set the value of a bit to 1 is dependent on properties of the underlying transistors, which are subject to variances in manufacturing.\cite{razafindraibe2006formal}
Minimal differences in clock timings between both paths can also reduce the security of \dual{}\cite{baddam2008path}.
Storing the inverse also requires significantly larger circuitry, doubling the circuit size or more\cite{baddam2008path}.

Even with these caveats, \dual{} has the major advantage that once it is applied, \emph{any} code can be run without modifications while still benefiting from the increased robustness.

\end{document}
